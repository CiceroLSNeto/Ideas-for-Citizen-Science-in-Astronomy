% ==============================================================================
%
% "Ideas for Citizen Science in Astronomy"
%
% ARAA, in preparation
%
% Copyright 2014 P.J.Marshall, C.J.Lintott & L.N.Fletcher 
%
% 17000 words, 80 references, 2 large figures: 35 pages
% from http://www.annualreviews.org/page/authors/article-length-estimator-2
%
% Justified to an editor window width of 80 characters 
% ==============================================================================

\documentclass{ar2e}


% ==============================================================================

\begin{document}

% ------------------------------------------------------------------------------

\jname{Annu.\ Rev.\ Astron.\ Astrophys.}
\jyear{2015}
\jvol{}
\ARinfo{}

\title{Ideas for Citizen Science in Astronomy}

\author{%
Philip J.\ Marshall
  \affiliation{Kavli Institute for Particle Astrophysics and Cosmology, P.O.~Box~20450, \newline
   MS~29, Stanford, CA 94309, USA.}
Chris J. Lintott
  \affiliation{Department of Physics, Denys Wilkinson Building, University of Oxford, \newline
   Keble Road, Oxford, OX1 3RH, UK.}
Leigh N. Fletcher
  \affiliation{Atmospheric, Oceanic and Planetary Physics, Clarendon Laboratory, University
   of Oxford, Parks Road, Oxford OX1 3PU}
}

\markboth{Marshall, Lintott \& Fletcher}{Citizen Astronomy}

% ------------------------------------------------------------------------------

% \begin{keywords}
% Go here...
% \end{keywords}

\begin{abstract} 

We review the relatively new, internet-enabled, and rapidly-evolving field of
citizen science, focusing on research projects in stellar, extragalactic and 
solar system astronomy that have benefited from the participation of members of
the public, often in large numbers. We find these volunteers making
contributions to astronomy in a variety of ways: making and analyzing new
observations, visually classifying features in images and light curves,
exploring models constrained by astronomical datasets, and initiating new
scientific enquiries.  The most productive citizen astronomy projects involve
close collaboration between the professionals and amateurs involved, and occupy
scientific niches not easily filled by great observatories or machine learning
methods: citizen astronomers are most strongly motivated by being of service to
science. In the coming years we expect participation and productivity in citizen
astronomy to increase, as survey datasets get larger and citizen science
platforms become more efficient. Opportunities include engaging the public in
ever more advanced analyses, and facilitating citizen-led enquiry by designing
professional user interfaces and analysis tools with citizens in mind.

\end{abstract}

\maketitle

\end{document}

% ==============================================================================
