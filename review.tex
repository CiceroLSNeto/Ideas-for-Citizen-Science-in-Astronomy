% ============================================================================
%
% "Ideas for Citizen Science in Astronomy"
%
% Marshall, Lintott & Fletcher, ARAA (2014)
%
% ============================================================================

\documentclass{ar2e}

\usepackage{ulem}
\usepackage{ARAstroBib}
\usepackage{amssymb,amsbsy,psfig}
\usepackage{xspace}
\usepackage[usenames]{color}

% JOURNALS
\def\apj{ApJ}                                         
\def\apjs{ApJS}
\def\apjl{ApJL}
\def\aap{A{\&}A}
\def\aaps{A{\&}AS}
\def\mnras{MNRAS}
\def\aj{AJ}
\def\araa{ARAA}
\def\pasp{PASP}
\def\nat{Nature}
\def\prd{Phys.\ Rev.\ D}

% MISC
\def\eg{{\it e.g.}\xspace}
\def\ie{{\it i.e.}\xspace}
\def\cf{{\it c.f.}\xspace}
\def\etal{et~al.\xspace}

% CROSS-REFERENCES
\def\Sref#1{Section~\ref{#1}\xspace}
\def\Fref#1{Figure~\ref{#1}\xspace}
\def\Tref#1{Table~\ref{#1}\xspace}
\def\Eref#1{Equation~\ref{#1}\xspace}
\def\Eqref#1{Eq.~(\ref{#1})\xspace}

% COMMENTING
\newcommand{\phil}[1]{\textcolor{blue}{\bf PJM: #1}}
\newcommand{\chris}[1]{\textcolor{blue}{\bf CJL: #1}}
\newcommand{\leigh}[1]{\textcolor{blue}{\bf LDF: #1}}
\newcommand{\todo}[2]{{\bf \it TODO: #1: #2}}
\newcommand{\query}[2]{{\it \textcolor{red}{Q: #1: #2}}}
\newcommand{\answer}[2]{{\it \textcolor{blue}{A: #1: #2}}}


% ============================================================================

\begin{document}

% ----------------------------------------------------------------------------

\jname{Annu.\ Rev.\ Astron.\ Astrophys.}
\jyear{2014}
\jvol{}
\ARinfo{}

\title{Ideas for Citizen Science in Astronomy}

% Provisional author list:
\author{Phil Marshall,
Chris Lintott, and
Leigh Fletcher
\affiliation{Department of Physics, Denys Wilkinson Building, 
University of Oxford, Keble Road, Oxford,
OX1 3RH, UK.}}

\markboth{Marshall, Lintott \& Fletcher}{Citizen Science in Astronomy}

% ----------------------------------------------------------------------------

% \begin{keywords}
% Go here...
% \end{keywords}

\begin{abstract}
Write abstract here.
\end{abstract}

\maketitle

% ============================================================================

\section{Introduction}
\label{sec:intro}

Define ``citizen'' as opposed to ``professional.'' Define science as process of
enquiry. Divide this process into steps, consider each. 

% Defs are at heart of answer to FAQ: is CS science or outreach/education? 

Importance of the internet. Why this review now? CS exploded because people are 
now networked, both citizens and professionals. Even offline science is
connected by web.

Aim to review the literature, for ideas  for astronomy. Much to learn from other
fields. Review all of  citizen science with ideas for astronomy in mind.

Ideas FOR astronomy: we will look both backward and forward, and try to answer:
How can the full potential of citizen science be realised, in astronomy? 

Niches for CS?
Benefits to science from CS?
Benefits to society from CS?
CS in the context of the Big Data problem.

Where should ideas for the future come from? Which community?
Explain methodology, and reasoning. Idea gathering by blog. Open writing.

Organisation of review: start with the community, and understand the citizenry.
Then return to science as process of enquiry, and divide into parts. Astronomy
typically starts with observations, follow this.

How are citizens already contributing in each area? Part 1: Ideas past and
present.  Observations, data analysis, citizen enquiry.
\Sref{sec:observers}--\ref{sec:builders} Summarize in \ref{sec:guidelines}

How might citizens contribute to science in the future?
Part 2: Ideas for the future. \Sref{sec:future}

Limits to citizen science. Challenges: data rates, complexity, collaboration,
accessibility. \Sref{sec:limits}

Concluding remarks. \Sref{sec:conclusions}


% ----------------------------------------------------------------------------

\section{Understanding the Citizens}
\label{sec:crowd}

Community first.

Raddick et al paper, others: motivations of users. 
Who is participating in citizen science? Who could be?

Link to schools: kids as citizens. 

Motivations.

Benefits to Citizens. Ethics of CS.
People as ends in themselves. Understanding of black boxes. 


% ----------------------------------------------------------------------------

\section{Data Acquisition: Citizen Observers}
\label{sec:observers}

* Barentsen meteor showers by twitter

* Submitted amateur observations

* Unconscious data generation - astrometry.net comet holmes

Biology projects for comparison?


% ----------------------------------------------------------------------------

\section{Citizen Instrumentation}
\label{sec:builders}

Hardware development: Jansky. Recent examples? Rosing.

Software development: Stumm at astrometry.net.  Collaborative development
projects with citizens. PlanetHunters, Galaxy Zoo analysis.


% ----------------------------------------------------------------------------

\section{Data Processing: Citizen Classifiers}
\label{sec:classifiers}

* Galaxy Zoo

* Moon Zoo, Moonwatch

* Planethunters


% ----------------------------------------------------------------------------

\section{Data Modeling: Citizen Analysts}
\label{sec:modelers}

* Image processing, Jupiter. Mars rovers.

* Neptune encounter

* Milky Way Project

* Planethunters, offline analysis

Mergers, Foldit. Gamification. 

% ----------------------------------------------------------------------------

\section{Data Exploration: Citizen Enquiry}
\label{sec:inquirers}

Free-form investigation led by citizens.

Individuals in action:

Teacher-led science: bees. 

Monster eyes:
http://blogs.discovermagazine.com/notrocketscience/2012/10/30/12-year-old-uses-dungeons-and-dragons-to-help-scientist-dad-with-his-research/

Supported by online platforms:

* Voorwerp, Green Peas, enabled by forum.

* Planet Hunters examples

* Deep sky observers, variable nebulae etc. Amateur asteroids.

% ----------------------------------------------------------------------------

\section{Summary: Characteristics of Successful Citizen Science}
\label{sec:guidelines}

ie guidelines for future users.


% ----------------------------------------------------------------------------

\section{Ideas for the future}
\label{sec:future}

Preamble.

\subsection{Methodology}
\label{sec:future:method}

Crowd-sourced idea generation. 
Invite contributions from everywhere. Track and
analyze. 
Publicise to dotastronomy, facebook astronomers, twitterverse, wider still.
Foreign countries via google translate.
Organise and critically review submissions, and submitters. Selection bias.
Who is motivated? Who self-selected? 

% - - - - - - - - - - - - - - - - - - - - - - - - - - - - - - - - - - - - - - - 

\subsection{Observations in the future}
\label{sec:future:obs}

Harnessing Telescopes - robotic remote systems for amateurs

Eyes on the Sky - low cost photography

% - - - - - - - - - - - - - - - - - - - - - - - - - - - - - - - - - - - - - - - 

\subsection{Instrumentation in the future}
\label{sec:future:instr}

Where are observing communities headed? Where could they be taken?

% - - - - - - - - - - - - - - - - - - - - - - - - - - - - - - - - - - - - - - - 

\subsection{Classification in the future}
\label{sec:future:class}

Live data: task assignment. 

Human-computer partnerships. Replacing citizens, see SN Zoo.

% Citizen access to crowd results?

% - - - - - - - - - - - - - - - - - - - - - - - - - - - - - - - - - - - - - - - 

\subsection{Data modelling in the future}
\label{sec:future:models}

Easily inatslled apps or browser-based tools enable outsourcing of data
modelling. Operation of code, development of code.


% - - - - - - - - - - - - - - - - - - - - - - - - - - - - - - - - - - - - - - - 

\subsection{Scientific enquiry in the future}
\label{sec:future:enquiry}

Huge public databases. User interfaces designed for anyone. Enable social
networks. Provide publishing support, see Letters.


% ------------------------------------------------------------------------------

\section{The Limits of Citizen Science}
\label{sec:limits}

% - - - - - - - - - - - - - - - - - - - - - - - - - - - - - - - - - - - - - - - 

\subsection{Limiting data rates: some worked examples}
\label{sec:limits:data}

Large samples of lenses.  Transients with SKA.  Something from
planets. "Data rates". 

% - - - - - - - - - - - - - - - - - - - - - - - - - - - - - - - - - - - - - - - 

\subsection{Limits from complexity}
\label{sec:limits:complexity}

Difficult analyses. Microtasking only suitable for certain parts of the process?

% - - - - - - - - - - - - - - - - - - - - - - - - - - - - - - - - - - - - - - - 

\subsection{Limits to collaboration}
\label{sec:limits:collab}

Collaboration between professional and citizen astronomers. Does it scale?
Communication issues: forum, letters. Contrast supervisor to student, with
scientist to crowd. Prospects for large collaborations? Collaborations between
citizens, eventually linked to professionals?

% - - - - - - - - - - - - - - - - - - - - - - - - - - - - - - - - - - - - - - - 

\subsection{Limited access}
\label{sec:limits:access}

Connection to citizen science: open data, open publishing. Citizens reading
papers. Accessibility. Potential barriers. International CS. Language barriers, 
cultural issues. 

% ------------------------------------------------------------------------------

\section{Concluding Remarks}
\label{sec:conclusion}

Astronomy's special place in science. Shared curiosity about the night sky. Most
accessible of mysteries. The population of Earth who
are curious about the night sky. 

% ----------------------------------------------------------------------------
Here is a test sentence, with a test citation \citep{Lin++08}.

\section{LITERATURE CITED}

% ARAA style:
\bibliographystyle{Astronomy}

% Use bibtex:
\bibliography{references}

% ----------------------------------------------------------------------------

\end{document}

% ============================================================================
