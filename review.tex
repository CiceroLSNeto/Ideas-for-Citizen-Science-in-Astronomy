% ==============================================================================
%
% "Ideas for Citizen Science in Astronomy"
%
% Marshall, Fletcher, & Lintott, ARAA (2014)
%
% ==============================================================================

\documentclass{ar2e}

\usepackage{ulem}
\usepackage{ARAstroBib}
\usepackage{amssymb,amsbsy,psfig}
\usepackage{xspace}
\usepackage[usenames]{color}

% JOURNALS
\def\apj{ApJ}                                         
\def\apjs{ApJS}
\def\apjl{ApJL}
\def\aap{A{\&}A}
\def\aaps{A{\&}AS}
\def\mnras{MNRAS}
\def\aj{AJ}
\def\araa{ARAA}
\def\pasp{PASP}
\def\nat{Nature}
\def\prd{Phys.\ Rev.\ D}

% MISC
\def\eg{{\it e.g.}\xspace}
\def\ie{{\it i.e.}\xspace}
\def\cf{{\it c.f.}\xspace}
\def\etal{et~al.\xspace}

% CROSS-REFERENCES
\def\Sref#1{Section~\ref{#1}\xspace}
\def\Fref#1{Figure~\ref{#1}\xspace}
\def\Tref#1{Table~\ref{#1}\xspace}
\def\Eref#1{Equation~\ref{#1}\xspace}
\def\Eqref#1{Eq.~(\ref{#1})\xspace}

% COMMENTING
\newcommand{\phil}[1]{\textcolor{blue}{\bf PJM: #1}}
\newcommand{\chris}[1]{\textcolor{blue}{\bf CJL: #1}}
\newcommand{\leigh}[1]{\textcolor{blue}{\bf LDF: #1}}
\newcommand{\todo}[2]{{\bf \it TODO: #1: #2}}
\newcommand{\query}[2]{{\it \textcolor{red}{Q: #1: #2}}}
\newcommand{\answer}[2]{{\it \textcolor{blue}{A: #1: #2}}}


% ==============================================================================

\begin{document}

% ------------------------------------------------------------------------------

\jname{Annu.\ Rev.\ Astron.\ Astrophys.}
\jyear{2014}
\jvol{}
\ARinfo{}

\title{Ideas for Citizen Science in Astronomy}

\author{Phil Marshall,$^{1,2}$
Leigh Fletcher,$^{2}$ and
Chris Lintott$^{2}$
\affiliation{%
\small
$^1$ Kavli Institute for Particle Astrophysics and Cosmology, P.O.~Box~20450, \newline
MS~29, Stanford, CA 94309, USA. \newline
$^2$ Department of Physics, Denys Wilkinson Building, University of Oxford, \newline
Keble Road, Oxford, OX1 3RH, UK.}}

\markboth{Marshall, Lintott \& Fletcher}{Citizen Science in Astronomy}

% ------------------------------------------------------------------------------

% \begin{keywords}
% Go here...
% \end{keywords}

\begin{abstract} 

We review the relatively new, internet-enabled, and rapidly evolving field of
citizen science, focusing on ideas from which astronomy either has benefited, or
could benefit in the future. We consider contributions to science in the form of
observations, instrumentation, data processing, data modeling and the design of
new scientific inquiries. Engaging a large and diverse community of both
professionals and citizens, we digest and present their suggestions for ideas
for citizen astronomy in the future. The limits to this approach to scientific
investigation are not yet known, but we make some rough estimates for astronomy
in particular.

\end{abstract}

\maketitle

% ==============================================================================

\section{Introduction}
\label{sec:intro}

The term ``Citizen Science'' refers to the activities of people who are not paid
to carry out scientific research, but nevertheless make intellectual
contributions to scientific research in their spare time. These contributions
are diverse, both in type and research area. The people who make those
contributions can, and do, come from all walks of life. This review is about the
science projects they have participated in to date, the tasks they have
performed, and how astronomy has benefited -- and could benefit further -- from
their efforts.

Citizen involvement in science pre-dates the profession itself, and there is a
long and honourable tradition of amateur observers making important discoveries
and significant sustained contributions. However, the advent of the world wide
web has changed the face of professional and amateur collaboration, providing
new opportunities and accelerating the sharing of information. People are now
connected to each other in a way that has never happened before. Professional
scientists can interact with citizens via a range of web-based media, including
purpose-built citizen science websites which increase the potential for shared
data analysis and exploration as well as data collection. Meanwhile, communities
of citizens have sprung into existence as like-minded people have been able to
find and talk to each other in a way that is almost independent of their
geographical location. The result has been an exponential increase in citizen
involvement in science. The field is evolving very quickly, with more and more
professional scientists becoming aware of the possibilities offered by
collaborating with, for example, specialists operating outside the usual
parameters of professional astronomical observation, or tens of thousands of
people eager to perform analysis in their lunch hours via microtasking.  Our aim
in this review is to review the scientific literature as it stands for ideas
implemented in citizen science projects, primarily in astronomy but also in
other fields, and then produce a summary of successful project characteristics
for future research groups to learn from.

As our title states, this is a review of ideas for astronomy. We will look
forward as well as back, and try to answer the questions: How can the full
potential of citizen science be realised in astronomy? What are the particular
niches that citizen science can fill, in our field? How might it contribute to
the solutions of the Big Data problem in astronomy?

% Somewhat in the spirit of citizen science, we are crowd-sourcing these
% questions via a blog, \stellarpops, set up to accompany this article.  Each
% area we cover will be opened up for discussion in the comments section, and we
% are encouraging the leading proponents to join in that discussion. Likewise,
% we are encouraging involvement from the citizens already working in astronomy,
% and the  community of astronomers collaborating with them. The result will be
% an open conversation which we will distill, with appropriate acknowledgment,
% into these pages.
% 
% \BLOG{1}{Phil}{What's all this about?}

This review is organised as follows. We survey the contributions to science that
citizens have made to date, organized according to the stage of the scientific
enquiry that those contributions fell into. Astronomy research typically starts
with observations: so do we, in \Sref{sec:obs}. We then proceed through a
discussion of citizen instrumentation, data processing, data modeling and
finally citizen-led enquiry in 
Sections~\ref{sec:instr}--\ref{sec:explore}. With this overview in place,
we review in \Sref{sec:crowd} the literature on, and the collected experience
of, the population of citizens who have taken part, or are currently taking
part, in scientific research, before summarizing progress in citizen science to
date in \Sref{sec:summary}. In the second part of this review, we turn to the
future. We first report a variety of suggestions for how
citizens might contribute to astronomy there in \Sref{sec:future}. Then, in 
\Sref{sec:limits} we consider possible limits to citizen science, including
challenges associated with data rates and volumes, data complexity, the
difficulties of large-scale collaboration, and finally the barriers to
accessibility. Finally, we give some concluding remarks in
\Sref{sec:conclusions}.


% ------------------------------------------------------------------------------

\section{Data Acquisition: Citizen Observing}
\label{sec:obs}

Preamble.

% - - - - - - - - - - - - - - - - - - - - - - - - - - - - - - - - - - - - - - - 

\subsection{Active Observing (Leigh, Chris)}
\label{sec:obs:active}

% \BLOG{2}{Chris}{Are these the top 5 amateur contributions to astronomy?}

Short case studies:
\begin{itemize}
\item Impacts on planets, the Moon. 
\item International Meteoroid Association, world coverage.
\item Planetary observations: JUPOS observers, nightly monitoring. 
        Martian meteorology?
\item Cometary monitoring.
\item Asteroid and TNO searching.
\item Variable nebulae.
\item Supernova detection.
\end{itemize}

% \BLOG{3}{Leigh}{Will we see it coming? 
% How we might better monitor impacts on other worlds.}
% 
% \BLOG{??}{Leigh}{April Showers? 
% How Can Citizen Scientists Study Planetary Weather?}
% 
% \BLOG{??}{Leigh}{What would you like for Christmas, in 2020?  
% What instrumentation would you be asking Santa for?}


% - - - - - - - - - - - - - - - - - - - - - - - - - - - - - - - - - - - - - - - 

\subsection{Passive Observing (Phil)}
\label{sec:obs:passive}

Case studies:
\begin{itemize}
\item Lang et al: astrometry.net on flickr, the orbit of Comet Holmes.
\item Barentsen et al: detecting meteor showers with twitter.
\end{itemize}

% - - - - - - - - - - - - - - - - - - - - - - - - - - - - - - - - - - - - - - - 

\subsection{Data Aquisition in Other Fields (Chris, Phil)}
\label{sec:obs:non-astro}

Case studies:
\begin{itemize}
\item Ecology?
\item Social science?
\item Others?
\end{itemize}


% ------------------------------------------------------------------------------

\section{Citizen Instrumentation}
\label{sec:instr}

Observations typically made with citizens own equipment. In some cases this is
quite advanced, as a result of their own work. 

% - - - - - - - - - - - - - - - - - - - - - - - - - - - - - - - - - - - - - - - 

\subsection{Hardware development by citizens (Leigh)}
\label{sec:instr:hardware}

History: Grote Reber. Recent examples. Rosing at LCOGT. Spectroscopy.

% \BLOG{3}{Leigh}{Will we see it coming? 
% How we might better monitor impacts on other worlds.}
% 
% \BLOG{??}{Leigh}{April Showers? 
% How Can Citizen Scientists Study Planetary Weather?}
% 
% \BLOG{??}{Leigh}{What would you like for Christmas, in 2020?  
% What instrumentation would you be asking Santa for?}


% - - - - - - - - - - - - - - - - - - - - - - - - - - - - - - - - - - - - - - - 

\subsection{Software development by citizens (Chris, Phil)}
\label{sec:instr:software}

JUPOS measurers: wind measurement. 
Impact detection.

PlanetHunters, Galaxy Zoo analysis.

Stumm at astrometry.net.  

Collaborative development projects with citizens. 


% ------------------------------------------------------------------------------

\section{Data Processing}
\label{sec:class}

Preamble. Definition of ``data processing:'' reduction or compression, producing
summary statistics or descriptors, classifications. Relationship to data mining.
Data to knowledge (although not necessarily understanding).

Visual classification in astronomy. Historical tradition. Now largely web-based.
Concept of ``Zoo.'' 

Other types of basic data reduction? Distributed processing not counted, on the
grounds that citizens add no information themselves.

% - - - - - - - - - - - - - - - - - - - - - - - - - - - - - - - - - - - - - - - 

\subsection{Visual Classification in Astronomy (Chris)}
\label{sec:class:astro}

Astronomy has led the way here.

Case studies:
\begin{itemize}
\item Galaxy morphology with Galaxy Zoo
\item Surfaces of solar system bodies: Moon Zoo, Moonwatch. Saturn storms. JUPOS
measurers.
\item Time domain astronomy: Supernova Zoo, PlanetHunters
\item Rapid-reaction events (jovian/lunar impacts, storm/plume eruptions)
\item Data mining for asteroids and TNOs.
\end{itemize}


% - - - - - - - - - - - - - - - - - - - - - - - - - - - - - - - - - - - - - - - 

\subsection{Visual Classification in Other Fields (Chris)}
\label{sec:class:non-astro}

What else can we learn, from projects outside astronomy?


% - - - - - - - - - - - - - - - - - - - - - - - - - - - - - - - - - - - - - - - 

\subsection{Other Examples of Citizen Data Processing (Leigh)}
\label{sec:class:non-sensory}

What else have people been up to, turning raw data in to knowledge?

Case studies:
\begin{itemize}
\item Jupiter: lucky imaging
\item Mars rover images.
\end{itemize}

% \BLOG{3}{Leigh}{Will we see it coming? 
% How we might better monitor impacts on other worlds.}
% 
% \BLOG{??}{Leigh}{April Showers? 
% How Can Citizen Scientists Study Planetary Weather?}

% ------------------------------------------------------------------------------

\section{Data Modeling: Citizen Analysts}
\label{sec:model}

Preamble: interpreting reduced data, in the context of a model, leads to
understanding. The modeling part often has technical difficulties that computers
may find hard to overcome, associated with complex and/or computationally
expensive, likelihood functions.


% - - - - - - - - - - - - - - - - - - - - - - - - - - - - - - - - - - - - - - - 

\subsection{Data Modeling in Astronomy (Phil, Chris)}
\label{sec:model:astro}

Case studies:
\begin{itemize}
\item Neptune encounter
\item Image modeling: Milky Way Project
\item Lightcurve analysis: PlanetHunters' offline analysis
\item Galaxy Zoo mergers
\end{itemize}

% \BLOG{??}{Phil}{What's the most difficult thing a crowd can do?}


% - - - - - - - - - - - - - - - - - - - - - - - - - - - - - - - - - - - - - - - 

\subsection{Data Modeling in Other Fields (Phil, Chris)}
\label{sec:model:astro}

Case studies:
\begin{itemize}
\item Protein folding with Fold.it. Gamification as a technique/
\item Other examples? 
\end{itemize}

% \BLOG{??}{Phil}{Who's really doing all the work?}


% ------------------------------------------------------------------------------

\section{Data Exploration: Citizen Enquiry}
\label{sec:explore}

Preamble. Free-form investigation led by citizens. Small fraction of citizen
science activity, but highly instructive to study them. This is the area of
greatest potential.

% \BLOG{??}{Phil}{How important is it to talk?}

% - - - - - - - - - - - - - - - - - - - - - - - - - - - - - - - - - - - - - - - 

\subsection{Individuals in action (Phil, Chris)}
\label{sec:explore:individuals}

Case studies:
\begin{itemize}
\item Teacher-led science: bee behaviour. 
\item Families as research groups: Monster eyes. % http://blogs.discovermagazine.com/notrocketscience/2012/10/30/12-year-old-uses-dungeons-and-dragons-to-help-scientist-dad-with-his-research/
\end{itemize}


% - - - - - - - - - - - - - - - - - - - - - - - - - - - - - - - - - - - - - - - 

\subsection{Facilitated research groups (Chris, Phil)}
\label{sec:explore:groups}

Case studies:
\begin{itemize}
\item Galaxy Zoo forum. Voorwerp, Green Peas. Lens thread: search and model.
\item Planet Hunters' investigations
\item Deep sky obs (variable nebulae etc). Amateur asteroid observations and follow-up.
\end{itemize}

% ------------------------------------------------------------------------------

\section{Understanding the Citizens}
\label{sec:crowd}

Preamble. Who participates in citizen science, and what motivates them?

% \BLOG{??}{Chris}{What's driving these people?}


% - - - - - - - - - - - - - - - - - - - - - - - - - - - - - - - - - - - - - - - 

\subsection{Demographics (Chris, Phil, Leigh)}
\label{sec:crowd:demographics}

Who is participating in citizen science? Who could be, but is not?
Breakdown by activity, if possible.

Relationship between citizen science and schools, colleges. Education programmes
associated with citizen science.

% - - - - - - - - - - - - - - - - - - - - - - - - - - - - - - - - - - - - - - - 

\subsection{Motivation (Phil, Chris, Leigh)}
\label{sec:crowd:motive}

What motivates citizen scientists?

Raddick et al paper for online classification: contributing to science as number
one motivation. Secondary motives: discovery/legacy.

Benefits to citizens.  Hobby becomes a useful tool. Satisfaction comes from
working towards improving our understanding of the Universe.

Similarities with professionals. 

% - - - - - - - - - - - - - - - - - - - - - - - - - - - - - - - - - - - - - - - 

\subsection{Ethics (Chris, Phil, Leigh)}
\label{sec:crowd:ethics}

Relationships between citizens and professionals. Mostly one-way? Examples of
two-way interactions: zoo forum. Solar system monitoring, spacecraft support.

Breaking down of boundaries. Professionals are citizens when outside their own
field. Citizens turning professional.

People as ends in themselves. The need to understand black box systems --
especially if the box is full of people. 

% \BLOG{??}{Phil}{Are citizen scientists wasting their time?}

% ------------------------------------------------------------------------------

\section{Summary: Characteristics of Successful Citizen Science}
\label{sec:summary}

To emerge.


% ------------------------------------------------------------------------------

\section{Ideas for the future}
\label{sec:future}

Preamble.

\phil{Should these parts be folded into the sections above? This might make for
an easier to read article.}

% - - - - - - - - - - - - - - - - - - - - - - - - - - - - - - - - - - - - - - - 
% 
% \subsection{Brief Notes on Methodology}
% \label{sec:future:method}
% 
% Crowd-sourced idea generation.  Invited contributions from everywhere. 
% Publicise to dotastronomy, facebook astronomers, twitterverse, wider still.
% Track and analyze.
% 
% Organise and critically review submissions, and submitters. Selection bias.
% Who is motivated? Who self-selected? 
% 
% - - - - - - - - - - - - - - - - - - - - - - - - - - - - - - - - - - - - - - - 

\subsection{Observations and Instrumentation in the future}
\label{sec:future:obs}

Robotic or automated telescopes to feed data to amateur processors/users for
immediate analysis.  Long term baselines with the same 
instrument/calibration.

Global telescope networks for continuous monitoring.
Distributed stations and networks for stellar occultations by TNOs and KBOs. 
Mobile observing stations and international coordination?

Video monitoring for meteors from multiple interlinked stations for 3D
trajectory reconstruction.

Amateur observing follows professional:
\begin{itemize}
\item Deeper field for amateur observations of Uranus and Neptune, particularly
near-IR.
\item Visible-light and near-IR spectroscopy; long-term datasets, serious
photometry.  Calibration, calibration, calibration...
\item Advanced technologies such as AO for image stabilisation?
\end{itemize}

Adoption of uniform standards for amateur imaging to be provided to online
databases (already underway with PVOL).


% - - - - - - - - - - - - - - - - - - - - - - - - - - - - - - - - - - - - - - - 

\subsection{Classification in the future}
\label{sec:future:class}

Live data: task assignment. 

Human-computer partnerships. Replacing citizens, see SN Zoo.

% Citizen access to crowd results?

% - - - - - - - - - - - - - - - - - - - - - - - - - - - - - - - - - - - - - - - 

\subsection{Data modelling in the future}
\label{sec:future:models}

Easily installed apps or browser-based tools enable outsourcing of data
modelling. Operation of code, development of code. Crowd-sourcing of current
detailed analyses.


% - - - - - - - - - - - - - - - - - - - - - - - - - - - - - - - - - - - - - - - 

\subsection{Scientific enquiry in the future}
\label{sec:future:enquiry}

Huge public databases from wide field surveys: LSST, Euclid, SKA. User
interfaces designed for anyone, with social networking enabled. 

Provide publishing support, see Letters.


% ------------------------------------------------------------------------------

\section{The Limits of Citizen Science}
\label{sec:limits}

We have argued that a critical part of `citizen science' lies in the ability of
the amateur to make an authentic contribution to science. Earlier in this part
of the review, we looked forward to a richer future for such interaction, but in
this section we consider the potential limits and checks on citizen science. 

% - - - - - - - - - - - - - - - - - - - - - - - - - - - - - - - - - - - - - - - 

\subsection{Data limits data rates: some worked examples}
\label{sec:limits:data}

Problems presented by data volume, and data rates. 
Case studies: Large samples of lenses?  Transients with SKA? 


% - - - - - - - - - - - - - - - - - - - - - - - - - - - - - - - - - - - - - - - 

\subsection{Limits from complexity}
\label{sec:limits:complexity}

Difficult analyses. Microtasking only suitable for certain parts of the process?


% - - - - - - - - - - - - - - - - - - - - - - - - - - - - - - - - - - - - - - - 

\subsection{Limits to collaboration}
\label{sec:limits:collab}

Collaboration between professional and citizen astronomers. Does it scale?
Communication issues: forum, letters. Contrast supervisor to student, with
scientist to crowd. Prospects for large collaborations? Collaborations between
citizens, eventually linked to professionals?


% - - - - - - - - - - - - - - - - - - - - - - - - - - - - - - - - - - - - - - - 

\subsection{Limits to access}
\label{sec:limits:access}

Connections between citizen science and open data, and open publishing. 
Citizens reading papers: accessibility, potential barriers. 

International CS. Language barriers, cultural issues. 


% ------------------------------------------------------------------------------

\section{Concluding Remarks}
\label{sec:conclusions}

Does astronomy have any sort of special place in citizen science?


% ------------------------------------------------------------------------------

\section*{Acknowledgments}

% We are most grateful to everyone who took the time to deposit their two cents
% in the comments section of the \stellarpops blog, and in particular the
% following contributors whose ideas helped shape \Sref{sec:future} but were not
% cited by name in the text: XXX, YYY, ZZZ. 

PJM thanks the Royal Society for financial support in the form of a university
research fellowship. 
%
CJL is grateful...
%
LDF acknowledges...
% 
This work was supported by...

% ------------------------------------------------------------------------------
% Here is a test sentence, with a test citation \citep{Lin++08}.

\section{Literature Cited}

% ARAA style:
\bibliographystyle{Astronomy}

% Use bibtex:
\bibliography{references}

% ------------------------------------------------------------------------------

\end{document}

% ==============================================================================
